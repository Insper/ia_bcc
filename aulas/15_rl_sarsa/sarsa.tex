\documentclass{beamer}
\usepackage[utf8]{inputenc}

\usetheme{Madrid}
\usecolortheme{default}
\usepackage{color}
\usepackage{colortbl}

\usepackage{algorithmic}

\def\HiLi{\leavevmode\rlap{\hbox to \hsize{\color{yellow!50}\leaders\hrule height .8\baselineskip depth .5ex\hfill}}}

\begin{document}

\title{Algoritmo \textsc{Sarsa}} 
\author{Fabrício Barth}
\institute{Insper Instituto de Ensino e Pesquisa}
\date{Fevereiro de 2025}

\maketitle

\begin{frame}{\textsc{Sarsa}: on-policy}
	
	A regra para update da \textbf{Q-table} no algoritmo \textbf{Q-Learning} é: 
	
	\begin{equation}
	Q(s,a) \leftarrow Q(s,a) + \alpha [r +\gamma \max_{A'}{Q(s', A')} - Q(s,a)]
	\end{equation}

	A diferença entre o novo valor e a estimativa antiga é utilizada para atualizar a estimativa antiga. O algoritmo \textbf{Q-Learning} considera como o novo valor o valor máximo das possibilidades no novo estado $s'$: 
	
	\begin{equation}
	\max_{A'}{Q(s', A')}
	\end{equation}
	
\end{frame}


\begin{frame}{\textsc{Sarsa}: on-policy}
	
	\begin{itemize}
		
	\item No entanto, a ação realmente executada pelo agente pode não ser a ação que tem o valor máximo em $s'$ devido a função de escolha da ação ser baseada no valor de $\epsilon$ e ter características aleatórias.
	
	\item Por isso que o algoritmo \textbf{Q-Learning} é chamado de \textbf{off-policy}.   
\end{itemize}
	
\end{frame}


\begin{frame}{\textsc{Sarsa}: on-policy}
	
	O algoritmo \textsc{Sarsa} é chamado de  \textbf{on-policy} porque ele atualiza a \textbf{Q-table} da seguinte forma: 
	
	\begin{equation}
		Q(s,a) \leftarrow Q(s,a) + \alpha [r +\gamma Q(s', a') - Q(s,a)]
	\end{equation}
	
	o algoritmo \textsc{Sarsa} atualiza $Q(s,a)$ considerando a real ação $a'$ executada pelo agente em $s'$.
	
\end{frame}

\begin{frame}{\textsc{Sarsa}: on-policy}
\begin{algorithmic} 
	\STATE \emph{\textbf{function} Sarsa(env, $\alpha$, $\gamma$, $\epsilon$, $\epsilon_{min}$, $\epsilon_{dec}$, episódios)}
	\STATE inicializar os valores de $Q(s, a)$ arbitrariamente
	\FOR {todos os episódios}
	\STATE inicializar $s$ a partir de $env$
	\STATE \HiLi $a \leftarrow escolha(s, \epsilon)$
	\REPEAT
	\STATE $s', r \leftarrow$ executar a ação $a$ no $env$
	\STATE \HiLi $a' \leftarrow escolha(s', \epsilon)$
	\STATE \HiLi $Q(s,a) \leftarrow Q(s,a) + \alpha [r +\gamma Q(s', a') - Q(s,a)]$
	\STATE $s  \leftarrow s'$
	\STATE \HiLi $a  \leftarrow a'$
	\UNTIL {$s$ ser um estado final}
	\STATE \textbf{if} $\epsilon > \epsilon_{min}$ \textbf{then} $\epsilon \leftarrow \epsilon \times \epsilon_{dec}$
	\ENDFOR
	\STATE \textbf{return} Q
\end{algorithmic}	
\end{frame}


\begin{frame}{Atividade de implementação}
	
	\begin{alertblock}{Comparando \textsc{Q-learning} e \textsc{Sarsa}}
		O objetivo desta atividade é implementar o algoritmo \textsc{Sarsa} e compará-lo com o \textsc{Q-learning}.
	\end{alertblock}
	
	\begin{block}{Atividades}
	Siga o roteiro descrito em https://insper.github.io/rl/classes/05\_x\_sarsa/ \href{https://insper.github.io/rl/classes/05_x_sarsa/}
	{\beamergotobutton{Link}}
	\end{block}

\end{frame} 



%
% ################################ VERSAO 2023 REVISADA ATEH AQUI ########################
%

%\begin{frame}{Ambientes não-determinísticos}
%	
%	\begin{alertblock}{Ambientes não-determinísticos}
%		O objetivo desta atividade é desenvolver agentes capazes de atuar em ambientes não-determinísticos.
%	\end{alertblock}
%	
%	\begin{block}{Atividades}
%		Siga o roteiro descrito em src/parte3/parte3.md \href{https://github.com/fbarth/reinLearn/blob/main/src/parte3/parte3.md}
%		{\beamergotobutton{Link}}
%	\end{block}
%\end{frame}
%
%
%\begin{frame}{Ambiente competitivo com variável aleatória}
%	\begin{alertblock}{Jogos}
%		O objetivo desta atividade é utilizar e analisar o comportamento de agentes em ambientes competitivos com variável aleatória.
%	\end{alertblock}
%	
%	\begin{block}{Atividades}
%		Siga o roteiro descrito em src/parte4/parte4.md \href{https://github.com/fbarth/reinLearn/blob/main/src/parte4/parte4.md}
%		{\beamergotobutton{Link}}
%	\end{block}
%\end{frame}
%
%\begin{frame}{Ambiente contínuo}
%	
%	\begin{alertblock}{Espaço de estados contínuo}
%		O objetivo desta atividade é fazer uso de aprendizagem de reforço em ambientes com um espaço de estados contínuo.
%	\end{alertblock}
%	
%	\begin{block}{Atividades}
%		Siga o roteiro descrito em src/parte5/parte5.md \href{https://github.com/fbarth/reinLearn/blob/main/src/parte5/parte5.md}
%		{\beamergotobutton{Link}}
%	\end{block}
%\end{frame}
%
%
%%%\begin{frame}{Q-Learning com GridSearch}
%%%Faz sentido usar GridSearch para encontrar os melhores valores de $\alpha$ e $\gamma$? 
%%%\end{frame}
%
%
%
%\begin{frame}{Considerações Finais}
%\begin{itemize}
%	\item O algoritmo \textit{Q-Learning} pode ser utilizado por agentes que 
%	não tem conhecimento prévio sobre o problema.
%	
%	%\item Diversos autores já provaram que o algoritmo \textit{Q-Learning} converge 
%	%para a função correta $Q$ dentro de certas condições. Por exemplo, uma 
%	%delas é garantir que o agente avalie um par $Q(s,a)$ diversas vezes.  
%	
%	%\newpage
%	
%	\item \textit{Q-Learning} converge em ambientes 
%	\textbf{determinísticos} e \textbf{não-determinísticos}.
%	
%	\item Na prática, o algoritmo \textit{Q-Learning} necessita de muitas iterações de 
%	treinamento até convergir, inclusive para problemas que não tem um espaço de busca 
%	tão grande.   
%
%\pause 
%
%	\item \textbf{E quando o espaço de busca for muito grande?} 
%
%	%\item Usar \textit{Deep Learning} com \textit{Reinforcement Learning}! \emph{Assunto do 
%	%nosso próximo encontro!}
%
%	%\item Ler o capítulo \textit{18. Reinforcement Learning} até a seção 
%	%\textit{Implementing Deep Q-Learning} do livro \textbf{Hands-On Machine Learning 
%	%with Scikit-Learn, Keras, and TensorFlow, 2nd Edition} do Aurélien Géron.
%
%	%\item Referência adicional: https://deepmind.com/research/case-studies/alphago-the-story-so-far
%
%\end{itemize}
%\end{frame}
%
%
%\begin{frame}{O que vimos até o momento?}
%\begin{itemize}
%	\item o que é \textbf{Aprendizagem por Reforço} e os seus principais conceitos;
%	\item como desenvolver um agente autônomo usando o algoritmo \textbf{Q-Learning};
%	\item quais os aspectos positivos e negativos do algoritmo \textbf{Q-Learning}, e;
%	\item funcionamento da biblioteca \textbf{Gym}. 
%\end{itemize}
%\end{frame}

\begin{frame}{Material de \textbf{consulta}}
\begin{itemize}
  %\item Tom Mitchell. Machine Learning. McGraw-Hill, 1997.
  \item Richard S. Sutton and Andrew G. Barto. 2018. Reinforcement Learning: An Introduction. A Bradford Book, Cambridge, MA, USA. \textbf{Capítulo 6.4} 
  \item Rummery, G. A. \& Niranjan, M. (1994), On-Line Q-Learning Using Connectionist Systems (TR 166), Technical report, Cambridge University Engineering Department, Cambridge, England.
  %\item Projeto Gymnasium \href{https://gymnasium.farama.org/}{\beamergotobutton{Link}}
  %\item Projeto PettingZoo \href{https://pettingzoo.farama.org/}{\beamergotobutton{Link}}
  %\item https://deepmind.com/research/case-studies/alphago-the-story-so-far
  \item Aurélien Géron. Hands-On Machine Learning with Scikit-Learn, Keras, and TensorFlow, 2nd Edition, 2019. 
\end{itemize}
\end{frame}

\end{document}

